% 結言
% 議論について書くことが多い.様々

%  本研究では,指先のソフトセンサから得られる触覚データを学習に取り入れることで,視覚のみでは対応しきれない状況にも適応可能なシステムを構築し,キムタオルの把持タスクでその触覚の有用性を示すことができた.さらに,重力補償制御を実装した7自由度ロボットアームのテレオペレーションシステムを構築し,操作者の身体的負担を軽減して直感的で効率的な教示データ収集を可能にしたことで,1人で行うのが難しいタスクを高い成功率で達成した.\par

% 今後,この知見を活かして,双腕でのマニピュレーションタスクや,より複雑なタスクへの応用を検討する.特に,双腕ロボットシステムにおいては,片方のアームで物体を保持しながらもう片方のアームで操作を行う場面が多く,重力補償制御が操作者の負担軽減に大きく寄与する,あるいは腕1つで7自由度あるような重いロボットでは必須になることが予想される.また,触覚情報を活用することで,腕が2本になることにより必然的に増えるセルフオクルージョンの問題を緩和し,より精密な操作が可能になると考えられる.



 本研究では,接触を伴う物体操作タスクにおいて,視覚情報のみでは捉えきれない環境変動への適応と,複雑なタスクにおける高品質な教示データ収集の実現を目的とし,イオン液体センサ内蔵ソフトフィンガーを用いた模倣学習手法および,重力補償制御を実装したテレオペレーションシステムを提案した.

 第一に,ソフトフィンガーの柔軟性と触覚センシングを統合した模倣学習の有効性について検証した.キムタオル把持タスクを用いた評価実験の結果,提案手法は視覚的なオクルージョンがあっても,接触時のセンサ応答に基づき把持動作を適応的に調整可能であることを示した.特に,視覚のみでは高さ推定が困難な未知の積層枚数条件において,触覚統合モデルが高い成功率を維持したことは,ソフトフィンガーによる形状適合性と内部流路の変形計測が,タスクのロバスト性向上に寄与することを実証するものである.一方で,学習された方策の汎化能力は教示データの運動学的範囲に依存しており,教示時の可動域を大きく逸脱するような幾何学的制約(極端に低い位置での把持など)に対しては,触覚情報のみでの補正に限界があることも確認された.\par

 第二に,重力補償制御を導入した教示システムの有用性について検証した.7自由度ロボットアームを用いたテレオペレーションにおいて,重力補償による操作支援は,アームの自重による操作者の身体的・認知的負荷を軽減するだけでなく,教示作業の中断と再開をシームレスに行うことを可能にした.折り紙タスクの実機実験では,この特性を活かし,操作者がロボット操作と治具への固定補助を交互に行う教示作業を円滑に遂行できることを確認した.\par

 今後の展望として,本研究で得られた知見を双腕マニピュレーションへと拡張することが挙げられる.双腕作業においては,片腕で対象を保持しつつ他方の腕で操作を行うといった,より複雑な協調動作が求められるため,重力補償による操作支援の重要性はさらに増大すると考えられる.また,触覚情報を活用することで,腕が2本になることにより必然的に増えるセルフオクルージョンの問題を緩和し,より精密な操作が可能になると考えられる.

\par