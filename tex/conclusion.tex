% 結言
% 議論について書くことが多い.様々
%  本研究では,重力補償制御を実装した7自由度ロボットアームのテレオペレーションシステムを構築し,操作者の身体的負担を軽減して直感的で効率的な教示データ収集を可能にした.さらに,指先のソフトセンサから得られる触覚データを学習に取り入れることで,視覚のみでは対応しきれない状況にも適応可能なシステムを構築した.提案手法の有効性を検証するため,紙を折るなどの複雑なタスクを対象に模倣学習実験を行い,重力補償制御と触覚情報の有用性を確認した.\par
%  以上より,本研究は模倣学習で自重の重いロボットアームを使用して複雑なタスクを実行させる際の重力補償制御の重要性と,触覚情報の有効性を示した.

% 本研究で得られた知見は,今後双腕のロボットシステムやより複雑なタスクへの応用が期待される.特に,双腕ロボットシステムにおいては,片方のアームで物体を保持しながらもう片方のアームで操作を行う場面が多く,重力補償制御が操作者の負担軽減に大きく寄与することが予想される.また,触覚情報を活用することで,腕が2本になることにより必然的に増えるセルフオクルージョンの問題を緩和し,より精密な操作が可能になると考えられる.\par