% 結言
% 議論について書くことが多い.様々

 本研究では,指先のソフトセンサから得られる触覚データを学習に取り入れることで,視覚のみでは対応しきれない状況にも適応可能なシステムを構築し,キムタオルの把持タスクでその触覚の有用性を示すことができた.さらに,重力補償制御を実装した7自由度ロボットアームのテレオペレーションシステムを構築し,操作者の身体的負担を軽減して直感的で効率的な教示データ収集を可能にしたことで,1人で行うのが難しいタスクを高い成功率で達成した.\par

今後,この知見を活かして,双腕でのマニピュレーションタスクや,より複雑なタスクへの応用を検討する.特に,双腕ロボットシステムにおいては,片方のアームで物体を保持しながらもう片方のアームで操作を行う場面が多く,重力補償制御が操作者の負担軽減に大きく寄与する,あるいは腕1つで7自由度あるような重いロボットでは必須になることが予想される.また,触覚情報を活用することで,腕が2本になることにより必然的に増えるセルフオクルージョンの問題を緩和し,より精密な操作が可能になると考えられる.