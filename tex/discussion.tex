% 考察・検証
\subsection{触覚情報の活用による把持成功率の向上}
 キムタオルの把持タスクにおいて触覚情報の有無による学習モデルの比較検討を行った結果,教示データと同じ指の長さでの把持タスクにおいては,触覚ありモデルの方は教示データに含まれていない高さでも100\%の成功率を達成した.一方,触覚なしモデルでは把持高さが教示データと違うものでは成功率が大きく低下した上,教示データと同じ高さでも一部成功率の低下が確認された.この原因として,まず視覚情報の限界があげられうこれはそもそも横からのカメラで高さが分からず,上からのカメラもDepth情報のないRGBカメラであるため,視覚情報だけでは把持高さの正確な推定が困難であったことが原因と考えられる.また,高さによって大きな成功率のばらつきがあるのは,上からのRGBカメラのみによる視覚情報による高さの推定が,うまくできているものとできていないもののばらつきがあり,成功しているものは高さ推定の誤差をそもそもの指の柔らかさが吸収しているためと推察される.このカメラによる誤差を含む情報を触覚情報が補完することで,触覚込みの学習では高さに依らず高い成功率を達成できたと考えられる.
\par
触覚なしでうまくいったのにばらつきがある原因
視覚情報がバラバラで,うまくいっているものは,上からのrgbカメラのみで高さ判定が成功しており,うまくいっていないものは高さ判定が失敗しているように思う.画像で比べてみると,うまくいっているものは教示データと似たような見え方をしている.



また,教示データと異なる指の長さでの把持タスクにおいても,触覚ありモデルでは高い成功率が維持された.ただ,教示データ時よりも短い指の場合,触覚ありモデルでも1枚のキムタオルの時はつかめなかった.これはそもそもそこまで深くロボットアームを動かすデータが教示データに含まれないため,柔らかさや触覚で適応的に掴む限界がここにあると推察される.触覚なしモデルの方は,指が短くなったことにより,掴みに行く深さが変わり,むしろ多めの枚数で成功が多くみられるため,教示データと同じ指の場合が多めの枚数の時深くつかみに行き過ぎていたと考えられる.\par

これらの結果から,触覚情報を用いることで視覚情報だけでは捉えきれないコンテクスト情報を補完でき,環境変化に対するロバスト性が向上することが示された.特に,物体の柔らかさや把持高さなどの把持に重要な要素を触覚情報が補完することで,視覚情報だけでは困難な把持タスクにおいても高い成功率を達成できることが分かった.\par


\subsection{重力補償制御によるタスクの幅の拡大}
 重力補償がない状態での教示データ収集においては,常にどちらかの手がロボットアームの自重を支えている必要があり,操作しながら考えなければいけないことが増える上,片手での操作はほぼ不可能であった.一方,重力補償制御を導入したことで,ロボットアームの重量を感じさせない無重力のような直感的な操作感が得られ,片手での操作が可能となった.そのうえ,ロボットアームをある位置に固定することが容易になり,固定した状態で人間がロボットアームから手を離してロボットと協働してタスクを補助することができるようになった.それを利用して,今回のような折り紙タスクにおいて,アームに折り紙を折らせて途中で人間が折り紙の固定を補助するという複雑な動作が可能になった.\par
定量的に重力補償の有無による操作者の負担等を比較することはできなかったが,実機を通して重力補償の有無を経験することで,教示データ収集時の操作者の主観的な負担が軽減されたことは明らかであった上,操作の幅が広がるということが分かった.\par

