% 考察・検証
% \subsection{触覚情報の活用による把持成功率の向上}
%  キムタオルの把持タスクにおいて触覚情報の有無による学習モデルの比較検討を行った結果,教示データと同じ指の長さでの把持タスクにおいては,触覚ありモデルの方は教示データに含まれていない高さでも100\%の成功率を達成した.一方,触覚なしモデルでは把持高さが教示データと違うものでは成功率が大きく低下した上,教示データと同じ高さでも一部成功率の低下が確認された.

% この原因として,まず視覚情報の限界があげられうこれはそもそも横からのカメラで高さが分からず,上からのカメラもDepth情報のないRGBカメラであるため,視覚情報だけでは把持高さの正確な推定が困難であったことが原因と考えられる.また,高さによって大きな成功率のばらつきがあるのは,上からのRGBカメラのみによる視覚情報による高さの推定が,うまくできているものとできていないもののばらつきがあり,成功しているものは高さ推定の誤差をそもそもの指の柔らかさが吸収しているためと推察される.このカメラによる誤差を含む情報を触覚情報が補完することで,触覚込みの学習では高さに依らず高い成功率を達成できたと考えられる.
% \par
% Sensor\_off条件において成功率にばらつきが生じた要因として,RGBカメラのみを用いた視覚情報による高さ推定の不安定さが考えられる.トップビューカメラは深度情報を持たないため,積層されたキムタオルの高さ変化を画像特徴の変化として捉える必要があるが,照明条件やわずかな位置ずれにより,その特徴量は不安定となる.実際、Sensor\_off条件での成功エピソードの映像を確認すると,教示データと類似した把持位置にアームが到達していた.一方で失敗エピソードでは,アームが対象の手前で停止,あるいは過剰に押し込む動作が確認された.
% これに対し,Sensor\_on条件では,視覚情報による位置推定に誤差が含まれる場合でも,ソフトフィンガーが対象に接触した際の抵抗値変化をトリガーとして把持動作が調整されていることが示唆される.これは,視覚情報の不確実性を触覚情報が補完し,タスクのロバスト性を向上させた結果であると結論付けられる.\par


% また,教示データと異なる指の長さでの把持タスクにおいても,触覚ありモデルでは高い成功率が維持された.ただ,教示データ時よりも短い指の場合,触覚ありモデルでも1枚のキムタオルの時はつかめなかった.これはそもそもそこまで深くロボットアームを動かすデータが教示データに含まれないため,柔らかさや触覚で適応的に掴む限界がここにあると推察される.触覚なしモデルの方は,指が短くなったことにより,掴みに行く深さが変わり,むしろ多めの枚数で成功が多くみられるため,教示データと同じ指の場合が多めの枚数の時深くつかみに行き過ぎていたと考えられる.\par

% これらの結果から,触覚情報を用いることで視覚情報だけでは捉えきれないコンテクスト情報を補完でき,環境変化に対するロバスト性が向上することが示された.特に,物体の柔らかさや把持高さなどの把持に重要な要素を触覚情報が補完することで,視覚情報だけでは困難な把持タスクにおいても高い成功率を達成できることが分かった.\par

% この模倣学習のアーキテクチャは教示データにない動きからある程度まで逸脱した動きを触覚を入れることによってできるようになるが,教示データの関節角度になく,大きく逸脱したところは,触覚があっても限界がある.これについては,このアーキテクチャの限界であり,教えてない以上のことはそこまでロバストに(結果の右の表の成功率の表のセンサ有での1枚のとき0%のこと)はできないということ



% \subsection{重力補償制御によるタスクの幅の拡大}
%  重力補償がない状態での教示データ収集においては,常にどちらかの手がロボットアームの自重を支えている必要があり,操作しながら考えなければいけないことが増える上,片手での操作はほぼ不可能であった.一方,重力補償制御を導入したことで,ロボットアームの重量を感じさせない無重力のような直感的な操作感が得られ,片手での操作が可能となった.そのうえ,ロボットアームをある位置に固定することが容易になり,固定した状態で人間がロボットアームから手を離してロボットと協働してタスクを補助することができるようになった.それを利用して,今回のような折り紙タスクにおいて,アームに折り紙を折らせて途中で人間が折り紙の固定を補助するという複雑な動作が可能になった.\par
% 定量的に重力補償の有無による操作者の負担等を比較することはできなかったが,実機を通して重力補償の有無を経験することで,教示データ収集時の操作者の主観的な負担が軽減されたことは明らかであった上,操作の幅が広がるということが分かった.\par
\subsection{触覚情報の活用による把持成功率の向上}

 キムタオルの把持タスクにおいて,触覚情報の有無による学習モデルの比較検討を行った.その結果,教示データと同一の指長さを用いた条件において,触覚ありモデル(Sensor\_on)は教示データに含まれない高さ条件であっても100\%の成功率を達成した.一方で,触覚なしモデル(Sensor\_off)では,把持高さが教示データと異なる場合に成功率が著しく低下したほか,教示データと同一の高さであっても一部で成功率の低下が確認された.\par

 Sensor\_off条件において成功率の低下およびばらつきが生じた主要因として,視覚情報のみに依存した高さ推定の限界が挙げられる.本実験環境では,側面カメラは障害物によるオクルージョンが発生し,上面カメラは深度情報を持たないRGBカメラである.そのため,積層されたキムタオルの高さ変化を画像特徴の変化としてのみ捉える必要があるが,照明条件の変動やわずかな位置ずれにより,その特徴量は不安定となる.実際,Sensor\_off条件における失敗エピソードを確認すると,アームが対象の手前で停止する,あるいは過剰に押し込むといった動作が確認された.成功しているエピソードに関しては,偶発的に画像特徴による高さ推定が成功したか,あるいは指の柔軟性が推定誤差を吸収した結果であると推察される.\par

 これに対し,Sensor\_on条件では,視覚情報による位置推定に誤差が含まれる場合でも,高い成功率を維持した.これは,ソフトフィンガーが対象に接触した際の抵抗値変化をトリガーとして把持動作が調整されているためであることが示唆される.すなわち,視覚情報の不確実性を触覚情報が補完することで,環境変動(高さの変化)に対するロバスト性が向上したと結論付けられる.\par

 次に,指の長さを変更した条件(教示時より短い指)での結果について考察する.この条件においても,触覚ありモデルは概ね高い成功率を維持した.しかし,対象物が「1枚(高さ最小)」の場合においては,触覚ありモデルであっても把持に失敗(成功率0\%)した.
 この原因は,教示データに含まれる関節角度の分布にあると考えられる.教示データには,指が短くなった分だけアームを深く下ろすような動作データは含まれていない.本手法で用いたACTアーキテクチャは,触覚情報を利用することで接触時の微調整(局所的な適応)は可能であるが,教示された関節角度の分布から大きく逸脱するような大域的な軌道修正(教示時よりもさらに深くアームを下げる動作)までは生成できない.したがって,1枚の条件での失敗は,学習ベースの手法における汎化性能の限界を示していると言える.\par
 一方で,触覚なしモデルにおいて,指を短くした条件の方がむしろ成功する場合が見られた(多めの枚数時).これは,教示データと同一の指長さでは「深く掴みに行き過ぎていた(過剰な押し込みがあった)」動作が,指が短くなったことで偶然適切な深さになったためと考えられる.\par

 以上の結果より,触覚情報の統合は,物体の柔らかさや接触判定といった視覚のみでは捉えきれないコンテクスト情報を補完し,タスクの成功率向上に大きく寄与することが示された.ただし,その適応能力は教示データの運動学的範囲に依存し,物理的な接触のみでは補いきれない幾何学的な制約が存在することも確認された.

\subsection{重力補償制御によるタスクの幅の拡大}

 本研究で導入した重力補償制御下で,アームの重量を感じさせない直感的な操作感が実現した.さらに特筆すべき点は,アームを任意の位置で静止させ,その姿勢を維持できることである.これにより,操作者が一時的にアームから手を離し,環境側(対象物や治具)への介入を行うといった協働作業が可能となった.

 この利点は,実施した折り紙タスクにおいて顕著に確認された.本タスクでは,「アームによる折り動作」と「人間による治具への固定補助」を交互に行う必要がある.重力補償機能により,操作者はロボットの操作を中断してもその作業状態を維持できるため,一人での教示の幅を大きく拡大できた.\par
 本実験では筋電位計測による作業負荷計測やアンケート手法,教示にかかる時間比較などの定量的な負荷評価は行っていないものの,実機運用を通じて,重力補償の導入が教示者の主観的負担を大幅に軽減し,かつ一人で教示可能なタスクの幅を拡大することを確認した.
