% 考察・検証

\subsection{7自由度ロボットアームのテレオペレーションシステムについて}
本研究では,触覚とソフトフィンガーを備えた7自由度ロボットアームの模倣学習により,折り紙を折るという複雑なタスクを実現した.重力補償がない状態での教示データ収集においては,常に両手でロボットアームを支えている必要があり,片手での操作はほぼ不可能であった.一方,重力補償制御を導入したことで,ロボットアームの重量を感じさせない直感的な操作感が得られ,ある程度までなら片手での操作が可能となった.そのうえ,ロボットアームをある位置に固定することが容易になり,固定した状態で人間がロボットアームから手を離してロボットと協働してタスクを補助することができるようになった.定量的に重力補償の有無による操作者の負担を比較することはできなかったが,実機を通して重力補償の有無を経験することで,教示データ収集時の操作者の主観的な負担が軽減されたことは明らかであった上,操作の幅が広がるということが分かった.\par



\subsection{展望}
今後,この知見を活かして,双腕でのマニピュレーションタスクや,より複雑なタスクへの応用を検討する.特に,双腕ロボットシステムにおいては,片方のアームで物体を保持しながらもう片方のアームで操作を行う場面が多く,重力補償制御が操作者の負担軽減に大きく寄与することが予想される.また,触覚情報を活用することで,腕が2本になることにより必然的に増えるセルフオクルージョンの問題を緩和し,より精密な操作が可能になると考えられる.\par