% 緒言
  % 背景
  % (過去の研究)
  % 研究目的
  % (過去の研究)
  % 方法
  % 結果・結論
 近年,ロボットに複雑な作業を習得させる手法として,人間が実演したデータから方策を学習する模倣学習(imitation learning)が注目されている\cite{schaalLearningDemonstration1996,zhaoLearningFineGrainedBimanual2023,ravichandarRecentAdvancesRobot2020}.従来の数理モデルに基づく制御では記述が困難なタスクであっても,模倣学習を用いることで,多様な環境や対象物に対応可能な柔軟なスキルを獲得できる可能性がある\cite{ShanGenBairateraruZhiYuniJidukuMoFangXueXiniyoruSanCiYuanQuMianShikiDongZuonoXueXi2023,yamaneSoftRigidObject2024}.特に,接触を伴う物体操作タスクにおいては,視覚情報だけでなく,接触時の変形や力を扱うソフトロボット技術との融合が期待されている\cite{chenDataDrivenMethodsApplied2025,chiRecentProgressTechnologies2018,dahiyaTactileSensingHumans2010a,hanOverviewDevelopmentFlexible2017a,shintakeSoftRoboticGrippers2018}.このように,模倣学習による動作生成能力と,ソフトマテリアルによる環境適応性を統合することは,ロボットが実世界で汎用的なタスクを遂行する上で重要なアプローチとなる.\par

 模倣学習において触覚情報を活用する試みは近年活発化しており,例えばTactile ALOHAでは結束バンドの挿入といった微細な位置合わせ\cite{zhaoLearningFineGrainedBimanual2023}を,Haptic-Informed ACTでは壊れやすい物体の把持\cite{eljuriHapticInformedACTSoft2025}を,それぞれ触覚情報を統合することで実現している.しかし,従来の硬い触覚センサやグリッパでは,対象物の形状になじむことが難しく,不定形物や壊れやすい物体に対しては局所的な応力集中により損傷を与えるリスクがあった\cite{zhuDeformableFragileObject2025}.そこで本研究では,柔軟なシリコンゴム内部にイオン液体流路を埋め込んだソフトフィンガー\cite{ohashiSoftTactileSensors2023}を採用する.このフィンガーは,対象物に接触した際に自身の形状を柔軟に変形させることで,対象物を傷つけることなく優しく把持できる.この形状適合性と,内部の流路変形に基づく触覚センシングを組み合わせることで,視覚だけでは捉えきれない接触状態をタスク遂行に活用することを目的とする.\par

また,このような柔軟な指による繊細なマニピュレーションを学習させるためには,教示データ自体が高品質であることが求められる.人間の腕のように柔軟な動作が可能な7自由度ロボットアーム\cite{rosenHumanArmKinematics2005}は複雑なタスクに適している一方で,自由度が多い分自重が重く,ソフトフィンガーの特性を活かした繊細な接触動作を教示する上で,操作の直感性を損なう要因となり得る.そこで本研究では,主要な提案であるソフトフィンガーを用いた模倣学習を支える基盤として,リーダーアームに重力補償制御を導入したテレオペレーションシステムを構築した.これにより,教示者がアームの重量を感じることなく,自身の腕の延長のように直感的に操作することが可能となり,質の高い教示データの収集を実現している.\par

% 本研究では,この接触情報を安価に作成可能なイオン液体センサを内蔵したソフトフィンガーを用いることで,この触覚がどれだけタスクの実現に寄与できるかを検証する.

% ここで,複雑なマニピュレーションの実現には人間の腕のように柔軟な7自由度ロボットアームが有効である.しかし,多自由度アームは自重による教示時の操作負担が大きく,繊細な接触を伴う教示データの収集が困難である.
% そこで本研究では,模倣学習の教示システムにおいて,教示者の操作するロボットアームに重力補償制御を導入したうえで,操作しやすいグリッパを開発したことで,アームの質量を感じないうえに自分の腕のように直感的に操作できる感覚を実現した.\par

 提案手法の有効性を検証するため,キムタオルの把持タスクと紙を折るタスクを対象に実機で実験を行った.
キムタオルの把持タスクでは,ソフトフィンガーの柔軟性と触覚情報の統合により,視覚情報だけでは困難な状況下でも,環境変動に対してロバストな物体把持が高い成功率で学習可能であることを示した.
また,紙を折るタスクを通じ,重力補償を取り入れた教示システムが,複雑な工程を含むタスクにおいて操作者の負担を軽減し,円滑なデータ収集に寄与することを確認した.\par

 本論文の構成は以下の通りである.第2章では,ソフトフィンガーからの触覚を統合した模倣学習の枠組みと,それを支える重力補償付きテレオペレーションシステムについて述べる.第3章では,模倣学習の実験の内容と結果を示す.第4章では結果を基に考察する.最後に,第5章で本研究の結論と今後の展望を述べる。
\par