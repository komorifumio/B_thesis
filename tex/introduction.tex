% 緒言
  % 背景
  % (過去の研究)
  % 研究目的
  % (過去の研究)
  % 方法
  % 結果・結論
 近年,ロボットに複雑な作業を習得させる手法として,人間が実演したデータから方策を学習する模倣学習(imitation learning)が注目されている\cite{schaalLearningDemonstration1996,zhaoLearningFineGrainedBimanual2023,ShanGenBairateraruZhiYuniJidukuMoFangXueXiniyoruSanCiYuanQuMianShikiDongZuonoXueXi2023,yamaneSoftRigidObject2024}.従来の数理モデルに基づく制御では記述が困難なタスクであっても,模倣学習を用いることで,多様な環境や対象物に対応可能な柔軟なスキルを獲得できる可能性がある.特に,接触を伴う物体操作タスクにおいては,視覚情報だけでなく,接触時の変形や力を扱うソフトロボット技術との融合が期待されている\cite{chiRecentProgressTechnologies2018,dahiyaTactileSensingHumans2010a,hanOverviewDevelopmentFlexible2017a,shintakeSoftRoboticGrippers2018}.このように,模倣学習による動作生成能力と,ソフトマテリアルによる環境適応性を統合することは,ロボットが実世界で汎用的なタスクを遂行する上で重要なアプローチとなる.\par
%  模倣学習において触覚情報を活用する試みは近年活発化しており,例えばTactile ALOHAでは結束バンドの挿入といった微細な位置合わせ\cite{zhaoLearningFineGrainedBimanual2023}を,Haptic-Informed ACTでは壊れやすい物体の把持\cite{eljuriHapticInformedACTSoft2025}を,それぞれ触覚情報を統合することで実現している.本研究ではこれらの知見を発展させ,教示データの質とコストおよび環境変化へのロバスト性に着目する.具体的には,教示データの収集時に重力補償制御を実装することで,多自由度ロボットの操作者の身体的負担を軽減し,直感的で効率的なデータ収集を可能にする.さらに,指先のソフトセンサから得られる触覚データを学習に取り入れることで,視覚のみでは対応しきれない状況にも適応可能なシステムを構築する.\par



 模倣学習において触覚情報を活用する試みは近年活発化しており,例えばTactile ALOHAでは結束バンドの挿入といった微細な位置合わせ\cite{zhaoLearningFineGrainedBimanual2023}を,Haptic-Informed ACTでは壊れやすい物体の把持\cite{eljuriHapticInformedACTSoft2025}を,それぞれ触覚情報を統合することで実現している.ここで,接触を伴う複雑なマニピュレーションの実現には人間の腕のように柔軟な7自由度ロボットアーム\cite{rosenHumanArmKinematics2005}が有効である.だが,7自由度ロボットアームは7つのモータを使う以上必然的に重量が重くなり,ロボットへの教示データ収集時に操作者に負担を強いる.このような接触を伴う繊細な操作を学習させるためには,質の高い教示データの収集が不可欠であるが,ロボットアームが重さが質の高い接触データの収集を阻害する要因となり得る.\par

そこで本研究では,模倣学習の教示システムにおいて,教示者の操作するロボットアームに重力補償制御を導入することで,アームの質量を感じさせない直感的な操作感を実現し,微細な力加減や接触状態を含む良質な教示データの効率的な収集を可能にした.また,人間がロボットアームを自分の腕のように直感的に操作しやすくなるグリッパを開発した.\par
 提案手法の有効性を検証するため,紙を折るタスクを対象に実機で実験を行い,この模倣学習システムの有用性を確認した.定量的に重力補償の有無による操作者の負担を比較することはできなかったが,重力補償なしではできないタスクを高い成功率で実行することができ,接触を伴う物体操作タスクにおける本システムの有用性が示された.\par

 本論文の構成は以下の通りである.第2章では,模倣学習のためのロボットシステムの概観を示す.第3章では,今回開発した操作性に優れたリーダアームについて述べる.第4章では,模倣学習手法について説明する.第5章では,実機実験の設定と結果を示す.第6章では,実験結果に基づく考察を行う.最後に,第7章で本研究の結論と今後の展望を述べる。\
\par