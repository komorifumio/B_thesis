\abstract
% 摘要
  % 背景1~2文. 少し広めの内容(ソフトロボット全体の話など)
  % (過去の研究)
  % 研究目的1文. タイトルに対応.この後に関連論文述べると良い.
  % (過去の研究)
  % 方法2~3文.
  % 結果・結論2文(わかったこと,ある程度の考察).
 近年,複雑なタスクをロボットに実行させる方法として,人間のデモンストレーションから効率的に方策を学習可能な模倣学習が注目されている.特に接触を伴う物体操作タスクにおいては,視覚情報だけでなく,接触時の変形や力を扱うソフトロボット技術との融合が期待されている.\par
 本研究では,重力補償制御を実装した7自由度ロボットアームのテレオペレーションシステムを構築し,イオン液体センサを内蔵したソフトフィンガーから得られる触覚データを取り入れた模倣学習手法を提案する.提案手法の評価として,キムタオルの把持タスクと紙を折るタスクを対象に実機実験を行った.キムタオルの把持タスクの結果から触覚情報の統合により,柔軟物体の把持において高い成功率を維持できる適応的な動作を学習可能であることが示された.また,紙を折るタスクによって重力補償を取り入れたテレオペレーションシステムが,操作の中断を伴う運用において不可欠であることが確認された.\par