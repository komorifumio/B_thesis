\abstract
% 摘要
  % 背景1~2文. 少し広めの内容(ソフトロボット全体の話など)
  % (過去の研究)
  % 研究目的1文. タイトルに対応.この後に関連論文述べると良い.
  % (過去の研究)
  % 方法2~3文.
  % 結果・結論2文(わかったこと,ある程度の考察).
 近年,複雑なタスクをロボットに実行させる方法として,人間のデモンストレーションから効率的に方策を学習可能な模倣学習が注目されている.特に接触を伴う物体操作タスクにおいては,視覚情報だけでなく,接触時の変形や力を扱うソフトロボット技術との融合が期待されている.\par
 本研究では,重力補償制御を実装した7自由度ロボットアームのテレオペレーションシステムを教示の基盤として構築し,イオン液体センサを内蔵したソフトフィンガーの触覚データを取り入れた模倣学習手法を提案する.提案手法の評価として,キムタオルの把持タスクと紙を折るタスクを対象に実機実験を行った.キムタオルの把持タスクの結果から触覚情報の統合により,環境変動に対してロバストな柔軟物体把持を高い成功率で学習可能であることが示された.また,紙を折るタスクによって重力補償を取り入れた本システムが,操作の中断を伴う複雑な教示工程において重要であることが確認された.\par