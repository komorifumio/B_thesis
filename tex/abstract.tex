\abstract
% 摘要
  % 背景1~2文. 少し広めの内容(ソフトロボット全体の話など)
  % (過去の研究)
  % 研究目的1文. タイトルに対応.この後に関連論文述べると良い.
  % (過去の研究)
  % 方法2~3文.
  % 結果・結論2文(わかったこと,ある程度の考察).
 近年,複雑なタスクをロボットに実行させる方法として,人間のデモンストレーションから効率的に方策を学習可能な模倣学習が注目されている.特に接触を伴う物体操作タスクにおいては,視覚情報だけでなく,接触時の変形や力を扱うソフトロボット技術との融合が期待されている.本研究では,重力補償制御を実装した7自由度ロボットアームのテレオペレーションシステムを構築し,イオン液体センサを内蔵したソフトフィンガーから得られる触覚データを学習に取り入れることで,接触を伴う繊細なタスクを実行した.提案手法の有効性を検証するため,紙を折るタスクを対象に実機で実験を行い,この模倣学習ロボットシステムの有用性を確認した.\par