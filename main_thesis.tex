\documentclass[11pt]{jarticle}

% 数式
\usepackage{amsmath,amssymb} 

% 図
\usepackage[dvipdfmx]{graphicx,color} 
\usepackage{adjustbox}
\usepackage{here}
\usepackage{caption} 
\captionsetup{format=hang, labelformat=simple, labelsep=period}
\usepackage{subcaption}
\usepackage{cleveref}  

%表
\usepackage{makecell} %セルを定義する

%その他
\usepackage{bm}
\usepackage{url}
\usepackage{siunitx}
\sisetup{inter-unit-product=\ensuremath{{}\cdot{}}} % 単位間に "·" を入れる
\usepackage{cite}
\usepackage{comment}  %\begin{comment} と \end{comment} の間に記述した内容が全て無視される。
%\includecomment{comment}でコメントブロック
% \usepackage{slashbox}
\usepackage{placeins}

% 使う.styの指定
% thesis_KU_platex.sty の中の、\def\areyoumaster{0} の値を卒論なら0、修論なら1にする
\usepackage{thesis_platex}
% \usepackage{researchmeeting_platex}

% === 表紙 ===========================================================
\isfinal{1}

% === 文章情報 =======================================================
\title{触覚とソフトフィンガーを備えた7自由度ロボットアームの模倣学習}
\author{小森 文雄}
\supervisor{細田 耕 教授\\
            川節 拓実 講師}
\debater{中西 弘明 講師}
\deadline{令和8年2月}

% 図表のキャプションを英語表記に変更
\renewcommand{\figurename}{Fig}
\renewcommand{\tablename}{Table}

\begin{document}
\setlength{\abovecaptionskip}{0mm}

% === 表紙 ===========================================================
\titlepage

% === 摘要 ===========================================================
\titleinabs{触覚とソフトフィンガーを備えた7自由度ロボットアームの模倣学習}
\abstract
% 摘要
  % 背景1~2文. 少し広めの内容(ソフトロボット全体の話など)
  % (過去の研究)
  % 研究目的1文. タイトルに対応.この後に関連論文述べると良い.
  % (過去の研究)
  % 方法2~3文.
  % 結果・結論2文(わかったこと,ある程度の考察).
 近年,複雑なタスクをロボットに実行させる方法として,人間のデモンストレーションから効率的に方策を学習可能な模倣学習が注目されている.特に接触を伴う物体操作タスクにおいては,視覚情報だけでなく,接触時の変形や力を扱うソフトロボット技術との融合が期待されている.\par
 本研究では,重力補償制御を実装した7自由度ロボットアームのテレオペレーションシステムを教示の基盤として構築し,イオン液体センサを内蔵したソフトフィンガーの触覚データを取り入れた模倣学習手法を提案する.提案手法の評価として,キムタオルの把持タスクと紙を折るタスクを対象に実機実験を行った.キムタオルの把持タスクの結果から触覚情報の統合により,環境変動に対してロバストな柔軟物体把持を高い成功率で学習可能であることが示された.また,紙を折るタスクによって重力補償を取り入れた本システムが,操作の中断を伴う複雑な教示工程において重要であることが確認された.\par

% === 目次 ===========================================================
\newpage
\tableofcontents

% === 本文 ===========================================================
\section{緒言}

  % 緒言
  % 背景
  % (過去の研究)
  % 研究目的
  % (過去の研究)
  % 方法
  % 結果・結論
 近年,ロボットに複雑な作業を習得させる手法として,人間が実演したデータから方策を学習する模倣学習(imitation learning)が注目されている\cite{schaalLearningDemonstration1996,zhaoLearningFineGrainedBimanual2023,ShanGenBairateraruZhiYuniJidukuMoFangXueXiniyoruSanCiYuanQuMianShikiDongZuonoXueXi2023,yamaneSoftRigidObject2024}.従来の数理モデルに基づく制御では記述が困難なタスクであっても,模倣学習を用いることで,多様な環境や対象物に対応可能な柔軟なスキルを獲得できる可能性がある.特に,接触を伴う物体操作タスクにおいては,視覚情報だけでなく,接触時の変形や力を扱うソフトロボット技術との融合が期待されている\cite{chiRecentProgressTechnologies2018,dahiyaTactileSensingHumans2010a,hanOverviewDevelopmentFlexible2017a,shintakeSoftRoboticGrippers2018}.このように,模倣学習による動作生成能力と,ソフトマテリアルによる環境適応性を統合することは,ロボットが実世界で汎用的なタスクを遂行する上で重要なアプローチとなる.\par
%  模倣学習において触覚情報を活用する試みは近年活発化しており,例えばTactile ALOHAでは結束バンドの挿入といった微細な位置合わせ\cite{zhaoLearningFineGrainedBimanual2023}を,Haptic-Informed ACTでは壊れやすい物体の把持\cite{eljuriHapticInformedACTSoft2025}を,それぞれ触覚情報を統合することで実現している.本研究ではこれらの知見を発展させ,教示データの質とコストおよび環境変化へのロバスト性に着目する.具体的には,教示データの収集時に重力補償制御を実装することで,多自由度ロボットの操作者の身体的負担を軽減し,直感的で効率的なデータ収集を可能にする.さらに,指先のソフトセンサから得られる触覚データを学習に取り入れることで,視覚のみでは対応しきれない状況にも適応可能なシステムを構築する.\par



 模倣学習において触覚情報を活用する試みは近年活発化しており,例えばTactile ALOHAでは結束バンドの挿入といった微細な位置合わせ\cite{zhaoLearningFineGrainedBimanual2023}を,Haptic-Informed ACTでは壊れやすい物体の把持\cite{eljuriHapticInformedACTSoft2025}を,それぞれ触覚情報を統合することで実現している.ここで,接触を伴う複雑なマニピュレーションの実現には人間の腕のように柔軟な7自由度ロボットアーム\cite{rosenHumanArmKinematics2005}が有効である.だが,7自由度ロボットアームは7つのモータを使う以上必然的に重量が重くなり,ロボットへの教示データ収集時に操作者に負担を強いる.このような接触を伴う繊細な操作を学習させるためには,質の高い教示データの収集が不可欠であるが,ロボットアームが重さが質の高い接触データの収集を阻害する要因となり得る.\par

そこで本研究では,模倣学習の教示システムにおいて,教示者の操作するロボットアームに重力補償制御を導入することで,アームの質量を感じさせない直感的な操作感を実現し,微細な力加減や接触状態を含む良質な教示データの効率的な収集を可能にした.また,人間がロボットアームを自分の腕のように直感的に操作しやすくなるグリッパを開発した.\par
 提案手法の有効性を検証するため,紙を折るタスクを対象に実機で実験を行い,この模倣学習システムの有用性を確認した.定量的に重力補償の有無による操作者の負担を比較することはできなかったが,重力補償なしではできないタスクを高い成功率で実行することができ,接触を伴う物体操作タスクにおける本システムの有用性が示された.\par

 本論文の構成は以下の通りである.第2章では,模倣学習のためのロボットシステムの概観を示す.第3章では,今回開発した操作性に優れたリーダアームについて述べる.第4章では,模倣学習手法について説明する.第5章では,実機実験の設定と結果を示す.第6章では,実験結果に基づく考察を行う.最後に,第7章で本研究の結論と今後の展望を述べる。\
\par
  
\section{提案手法}

  % 提案手法
\subsection{模倣学習のためのロボットシステム概要}

\begin{figure}[htbp]
  \centering
  \includegraphics[width=1.0\textwidth]{tex/Image/imitationlearningsystem2.drawio.pdf}
  \caption{Overview of the proposed imitation learning system using a teleoperation setup and haptic information}
  \label{fig:robot_system}
\end{figure}
本章では,提案する重力補償を適用したテレオペレーションシステムと,模倣学習手法について述べる.Fig\ref{fig:robot_system}にシステムの構成図を示す.本システムは,リーダー・フォロワー構成のロボットアーム,フォロワーアーム先端についている2つのイオン液体センサを内蔵したソフトフィンガー,フォロワーアームを上からと横から観測するカメラ2台,およびこれらを制御・学習するためのPCから構成される.テレオペレーションによる教示データ収集時には,操作者がリーダーアームを操作し,フォロワーアームで物体操作を行う.収集した教示データを用いて,模倣学習によりフォロワーアームの動作モデルを学習する.学習モデルの入力として,カメラ画像,フォロワーアームの関節角度,およびソフトフィンガーからの触覚情報を用いる.学習後は,フォロワーアームが自律的に物体操作タスクを実行する.以下に,各構成要素について詳細に述べる.\par

\subsection{操作性に優れたリーダーアーム}


\subsubsection{リーダーアームのグリッパ設計}
 リーダーアームのグリッパは直感的に操作しやすいように,手のひらで握る用の棒状の把手と,人差し指と親指を差し込むための穴を設けた.これにより,手のひら全体でグリッパを保持しつつ,指先で細かい開閉操作を行うことが可能になる.

\begin{center}
  \begin{minipage}{0.48\textwidth}
    \centering
    \includegraphics[width=\linewidth]{tex/Image/openleadergripper.drawio.pdf}
    \captionof{figure}{Gripper in open state}
    \label{fig:opened_leadergripper}
  \end{minipage}\hfill
  \begin{minipage}{0.48\textwidth}
    \centering
    \includegraphics[width=\linewidth]{tex/Image/closedleadergripper.pdf}
    \captionof{figure}{Gripper in closed state}
    \label{fig:closed_leadergripper}
  \end{minipage}
\end{center}

また,グリッパを握ると自然とちょうど人間の前腕とリーダーアームの前腕リンクがFig2.4のように対応する.これにより操作者は自分の腕を動かす感覚でリーダーアームを操作できる.

\begin{center}
  \begin{minipage}{0.48\textwidth}
    \centering
    \includegraphics[width=\linewidth]{tex/Image/leaderarmwithme.drawio.pdf}
    \captionof{figure}{Leader arm with operator (overview)}
    \label{fig:leader_arm_with_me}
  \end{minipage}\hfill
  \begin{minipage}{0.48\textwidth}
    \centering
    \includegraphics[width=\linewidth]{tex/Image/leadergripperwithme.pdf}
    \captionof{figure}{Leader gripper held by operator}
    \label{fig:leader_gripper_with_me}
  \end{minipage}
\end{center}


\newpage
\subsubsection{リーダアームに重力補償を適用したテレオペレーションによるデータ収集}

本研究では,リーダー・フォロワー方式のテレオペレーションシステムとして,RT社製の7自由度ロボットアームCRANE-X7を2台使用する.マニピュレータの自由度は7自由度,グリッパの自由度は1自由度であり,全体として8自由度のロボットを使用した.1台をリーダーアームとして操作者が操作し,もう1台をフォロワーアームとして物体操作を行う.フォロワーアームのグリッパにはソフトフィンガーを取り付け,リーダーアームの動きに追従し物体操作を実行する.ソフトウェアは,Hugging Faceが提供する模倣学習ライブラリであるlerobot\cite{cadene2024lerobot}を用いたPython環境をベースとしている.\par
 リーダーアームの操作時,CRANE-X7の重量は操作者に負担をかけるため,テレオペレーション時のリーダーアームには重力補償制御を適用する.
重力補償制御では,ロボットアームの現在関節位置から各関節に作用する重力トルクを計算し,その逆方向のトルクを各関節に加えることで,アームの重量を打ち消す.lerobotはデータ収集から学習,評価までを統一的に扱える優れたフレームワークであるが,重力補償機能はPython実装では制御周期の安定性やリアルタイム性に課題があった.そこで本研究では,RT社の提供するCRANE-X7の重力補償のC++ライブラリをpybibd11を用いてlerobotのPython環境に組み込む実装を行った.これにより,最新の学習フレームワークの恩恵を受けつつ,C++による滑らかで高応答な重力補償制御を両立させている.結果として,操作者はリーダーアームの質量を感じることなく,直感的な教示動作を行うことが可能になった.

\subsubsection{重力補償トルクの算出}

本実装では,再帰的ニュートン・オイラー法を用いて,各関節に必要な重力補償トルクを算出した.
リンク $i$ の質量を $m_i$,リンク座標系における重心位置ベクトルを $\boldsymbol{c}_i$,親リンク $i-1$ からの回転行列を ${}^i\boldsymbol{R}_{i-1}$ とする.

まず,ベースリンク($i=0$)に対して重力加速度 $\boldsymbol{g}$ と逆向きの仮想的な加速度 $\ddot{\boldsymbol{p}}_0 = -\boldsymbol{g} = [0, 0, 9.8]^T \, \mathrm{m/s^2}$ を与える.
根本から手先に向かって,各リンクの並進加速度 $\ddot{\boldsymbol{p}}_i$ および重力による慣性力 $\hat{\boldsymbol{f}}_i$ を次式で伝播させる.
\begin{align}
    \ddot{\boldsymbol{p}}_i &= {}^i\boldsymbol{R}_{i-1} \ddot{\boldsymbol{p}}_{i-1} \\
    \hat{\boldsymbol{f}}_i &= m_i \ddot{\boldsymbol{p}}_i
\end{align}

次に,手先から根本に向かって,リンク間に作用する力 $\boldsymbol{f}_i$ とモーメント $\boldsymbol{n}_i$ のつり合いを計算する.
手先負荷がない場合,末端条件は $\boldsymbol{f}_{N+1} = \boldsymbol{0}, \boldsymbol{n}_{N+1} = \boldsymbol{0}$ とする.
\begin{align}
    \boldsymbol{f}_i &= {}^i\boldsymbol{R}_{i+1} \boldsymbol{f}_{i+1} + \hat{\boldsymbol{f}}_i \\
    \boldsymbol{n}_i &= {}^i\boldsymbol{R}_{i+1} \boldsymbol{n}_{i+1} + ({}^i\boldsymbol{p}_{i+1} \times {}^i\boldsymbol{R}_{i+1} \boldsymbol{f}_{i+1}) + (\boldsymbol{c}_i \times \hat{\boldsymbol{f}}_i)
\end{align}
ここで,${}^i\boldsymbol{p}_{i+1}$ はリンク $i$ 原点から子リンク $i+1$ 原点への位置ベクトルである.

最終的に,関節 $i$ の駆動軸ベクトル $\boldsymbol{a}_i$ 成分を取り出すことで,必要な補償トルク $\tau_i$ が得られる.
\begin{equation}
    \tau_i = \boldsymbol{a}_i^T \boldsymbol{n}_i
\end{equation}\par
このトルクを電流値へと変換し各関節のモータに与えることで,重力補償制御を実現している.トルクから電流値への変換係数は,実機に合わせて調整を行った.


\newpage
\subsection{触覚を統合した模倣学習}
\subsubsection{触覚センサを内蔵したソフトロボットフィンガー}
 フォロワーアームのハンド部分には,柔軟な把持と接触検知を実現するためにソフトフィンガーを装着する.このフィンガーは柔らかいシリコンゴムの本体と,PLAの爪部分でできており,内部にはイオン液体を封入した流路を持つ触覚センサが埋め込まれている.
センサ素子として用いるイオン液体は導電性を有するほか,不揮発性や難燃性,高い熱安定性といった物理的特性を持つ\cite{armandIonicliquidMaterialsElectrochemical2009}.
この液体を柔軟な流路内に封入することで,対象物への接触により流路が変形し,その断面積 $A$ や長さ $L$ が変化する.
この幾何学的な変化は,イオン液体固有の電気抵抗率 $\rho$ を用いて以下の式(\ref{eq:resistance})で表される抵抗値 $R$ の変化として検出される\cite{chossatSoftStrainSensor2013}.

\begin{equation}
  R = \rho \frac{L}{A}
  \label{eq:resistance}
\end{equation}

なお,本実験ではイオン液体として1-Butyl-3-methylimidazolium Bis(fluorosulfonyl)imide(関東化学株式会社製)を使用した.ただし,このイオン液体は温度によって抵抗値が多少変化するため,データ収集時と,モデルの評価時に毎回キャリブレーションを行った.


\subsubsection{ACTによる模倣学習}
\begin{figure}[htbp]
  \centering
  \includegraphics[width=1.1\textwidth]{tex/Image/ACT.drawio.pdf}
  \caption{Action Chunking with Transformers (ACT) architecture used for imitation learning.This model extends the conventional ACT framework by introducing distinct sensor embeddings.}
  \label{fig:act}
\end{figure}

本研究では,7自由度ロボットアームで接触を伴う複雑なタスクを学習するためにAction Chunking with Transformers (ACT) \cite{zhaoLearningFineGrainedBimanual2023}を基盤としたアーキテクチャを採用する.ACTはConditional Variational Autoencoder(CVAE)とTransformerを組み合わせたモデルであり,現在の観測から単一の行動ではなく,未来の数ステップ分のアクション系列(Action Chunk)を一括して予測することで,時系列データの長期的な依存関係を学習することに長けており,滑らかな動作生成が実現できる.\par
従来のACTに本アーキテクチャではセンサ値を新たなモダリティとして導入している点が特徴である.具体的には,関節角度と同様にセンサ値をトークン化し,CVAEエンコーダーおよびデコーダーの両方に対して,画像特徴量や関節情報と並列に入力することで,センサ情報を統合した動作生成を可能にしている.













\section{評価実験}

  % 実験結果
\subsection{触覚情報の有用性を示すキムタオル把持タスク}

\begin{figure}[htbp]
  \centering
  \begin{minipage}[t]{0.69\textwidth}
    \vspace{0pt}
    \centering
    \includegraphics[width=\linewidth]{tex/Image/experimentsetup.drawio.pdf}
  \end{minipage}\hfill
  \begin{minipage}[t]{0.31\textwidth}
    \centering
    \begin{minipage}[t]{\linewidth}
      \vspace{0pt}
      \centering
      \includegraphics[width=\linewidth]{tex/Image/lateralview.pdf}
      \caption*{CAM1}
    \end{minipage}
    \vspace{3mm}
    \begin{minipage}[t]{\linewidth}
      \centering
      \includegraphics[width=\linewidth]{tex/Image/topview.pdf}
      \caption*{CAM2}
    \end{minipage}
  \end{minipage}
  \caption{Experimental setup for the object-grasping task (left), and example camera views (right): Camera 1 provides a lateral view and Camera 2 provides a top view. An occluding obstacle is placed to simulate bimanual occlusion so that the target object (a Kim towel) is not visible from the lateral viewpoint.}
  \label{fig:experiment_setup_grasp}
\end{figure}


 本実験では,触覚情報の有用性を検証するために,触覚センサのデータを含んだ学習と含まない学習の2つの条件で,キムタオルを掴んで別の場所に離すタスクの成功率を比較した.ただし,将来的な双腕マニピュレーションでのもう一方のアームによるオクルージョン,あるいは箱の中に物体が入っていて上からしか見えない状態を想定し,アームの横からのカメラでキムタオルの高さを見ることができないような紺色の障害物を設置した.実験のセットアップをFig\ref{fig:experiment_setup_grasp}に示す.\par

\subsubsection{タスクの教示,学習条件}
教示データは,キムタオルをFig3.1においてある場所にある枚数置いておき,リーダーアームでその上から1枚だけとって紺色の障害物の左側に離す動作をテレオペレーションで行い収集した.この時,双腕や箱を想定した障害物により横からのカメラではキムタオルの高さが見えない.\par
キムタオルの高さは枚数を50枚,40枚,30枚,20枚,10枚の5段階でそれぞれ2回の教示データ,合計10回の教示データを収集した.教示データ収集時は,センサ値も取得しており,学習時は,その値を無視して学習するsensor\_off条件と,センサ値を使用して学習するsensor\_on条件の2つで学習を行った.\par
学習の条件は,CPUにAMD Ryzen 9 9950X,GPUにNVIDIA GeForce RTX 5080,メモリ64GBを搭載したPCを使用した.OSはUbuntu 24.04.3 LTS(Kernel 6.14)を用いた.学習のbatch sizeは8,学習ステップ数は20000ステップとした.



\subsubsection{タスクの評価と成功率}
タスクの成否は,キムタオルを1枚だけ掴んで,障害物の左側のどこでもいいので移動させることができたら成功とした.ただし,掴み損ねて何度もつかみに行こうとした場合は失敗とした.
検証時は教示時のキムタオルの枚数から増やしたものと減らした未知の枚数のものを用意し,全部で60枚,50枚,40枚,30枚,20枚,10枚,1枚の7段階でそれぞれ5回の試行を行い,成功率を算出した.\par
以下にそれぞれの枚数のキムタオルの様子を示す.\par

\begin{figure}[htbp]
  \centering
  \begin{minipage}[t]{0.13\textwidth}
    \centering
    \includegraphics[width=\linewidth]{tex/Image/1.pdf}
    \caption*{1}
  \end{minipage}\hfill
  \begin{minipage}[t]{0.13\textwidth}
    \centering
    \includegraphics[width=\linewidth]{tex/Image/10.pdf}
    \caption*{10}
  \end{minipage}\hfill
  \begin{minipage}[t]{0.13\textwidth}
    \centering
    \includegraphics[width=\linewidth]{tex/Image/20.pdf}
    \caption*{20}
  \end{minipage}\hfill
  \begin{minipage}[t]{0.13\textwidth}
    \centering
    \includegraphics[width=\linewidth]{tex/Image/30.pdf}
    \caption*{30}
  \end{minipage}\hfill
  \begin{minipage}[t]{0.13\textwidth}
    \centering
    \includegraphics[width=\linewidth]{tex/Image/40.pdf}
    \caption*{40}
  \end{minipage}\hfill
  \begin{minipage}[t]{0.13\textwidth}
    \centering
    \includegraphics[width=\linewidth]{tex/Image/50.pdf}
    \caption*{50}
  \end{minipage}\hfill
  \begin{minipage}[t]{0.13\textwidth}
    \centering
    \includegraphics[width=\linewidth]{tex/Image/60.pdf}
    \caption*{60}
  \end{minipage}
  \caption{Representative images of the Kim towel at each sheet count (1, 10, 20, 30, 40, 50, 60).}
  \label{fig:towel_counts}
\end{figure}


さらに,指先のソフトフィンガーの長さを教示時とは短くしたものでも評価を行った.1つ目は教示データと同じ状態,2つ目はそれよりも指の長さが30mm短い状態である.指の長さが長いものにすると,センサ値がない場合に指を強く押し付けすぎて危険なため30mm長いものは使用しなかった.\par

\begin{figure}[htbp]
  \centering
  \begin{minipage}[t]{0.45\textwidth}
    \centering
    \includegraphics[width=\linewidth]{tex/Image/30mmfing.pdf}
    \captionof{figure}{Fingers extended by 30mm (same as demonstration)}\label{fig:30mmfing}
  \end{minipage}\hfill
  \begin{minipage}[t]{0.45\textwidth}
    \centering
    \includegraphics[width=\linewidth]{tex/Image/0mmfing.pdf}
    \captionof{figure}{Fingers extended by 0mm}\label{fig:0mmfing}
  \end{minipage}
\end{figure}

以下の表に,各条件での成功率を示す.Table\ref{tab:success_rate_30mm}は指先を教示データと同じ長さにした場合,Table\ref{tab:success_rate_0mm}は指先を30mm短くした場合の成功率である.\par

\begin{table}[htbp]
\centering
\begin{minipage}[t]{0.48\textwidth}
  \centering
  \captionof{table}{Success rate with fingers extended by 30mm (same as demonstration data)}
  \label{tab:success_rate_30mm}
  \begin{tabular}{|l|l|l|}
  \hline
  sheets & sensor\_on & sensor\_off \\ \hline
  60     & 100.0(5/5) & 20.0(1/5)   \\ \hline
  50     & 100.0(5/5) & 60.0(3/5)   \\ \hline
  40     & 100.0(5/5) & 0.0(0/5)    \\ \hline
  30     & 100.0(5/5) & 100.0(5/5)  \\ \hline
  20     & 100.0(5/5) & 100.0(5/5)  \\ \hline
  10     & 100.0(5/5) & 100.0(5/5)  \\ \hline
  1      & 100.0(5/5) & 20.0(1/5)   \\ \hline
  \end{tabular}
\end{minipage}\hfill
\begin{minipage}[t]{0.48\textwidth}
  \centering
  \captionof{table}{Success rate with fingers extended by 0mm (different from demonstration data)}
  \label{tab:success_rate_0mm}
  \begin{tabular}{|l|l|l|}
  \hline
  sheets & sensor\_on  & sensor\_off \\ \hline
  60     & 100.0(5/5) & 100.0(5/5) \\ \hline
  50     & 100.0(5/5) & 80.0(4/5) \\ \hline
  40     & 100.0(5/5) & 100.0(5/5) \\ \hline
  30     & 100.0(5/5) & 100.0(5/5) \\ \hline
  20     & 80.0(4/5)  & 0.0(0/5)  \\ \hline
  10     & 100.0(5/5) & 0.0(0/5)   \\ \hline
  1      & 0.0(0/5)   & 0.0(0/5)   \\ \hline
  \end{tabular}
\end{minipage}
\end{table}




\newpage
\subsection{重力補償がなければできない折り紙タスク}

\begin{figure}[htbp]
  \centering
  \includegraphics[width=0.9\textwidth]{tex/Image/experiment_setup.drawio.pdf}
  \caption{Experiment setup (layout, observation cameras, and equipment configuration)}
  \label{fig:experiment_setup}
\end{figure}


 重力補償がなければできないタスクとして,折り紙1枚をソフトフィンガーでつかみ,折り曲げるタスクを行った.ロボットアーム1台で折り紙を折る動作を行うため,折り紙の左端は2点で固定する治具をつけ,折りたたんだ際にも人間がその2点にたたむ紙を刺すという補助を行った.この補助をするためには,操作者が一旦ロボットアームから手を離して折り紙をジグに刺す必要があるため,人間が2人以上いない限りは重力補償制御が不可欠である.\par


\subsubsection{タスクの教示,学習条件}
ポリシーは計10回のエピソードを取り,治具の位置は変えずにまず人間がだいたい紙の左上に2点刺すようにセッティングした.そのため,紙の位置は毎回微妙にずれている.
各エピソードで,操作者はリーダーアームを操作して折り紙を折り,一旦折ったらその状態でアームを重力補償下で固定し,紙を治具の2点に刺すという補助を行い,折り目をつける段階へと移行した.\par


学習の条件は,CPUにAMD Ryzen 9 9950X,GPUにNVIDIA GeForce RTX 5080,メモリ64GBを搭載したPCを使用した.OSはUbuntu 24.04.3 LTS(Kernel 6.14)を用いた.学習のbatch sizeは8,学習ステップ数は20000ステップとした.


また,センサ値は生データをそのまま使用すると,温度変化によって異なる値を示すため,キャリブレーションを行った.具体的には,各試行の開始時に5秒間静止した状態でセンサ値を取得し,その平均値を基準として各時刻のセンサ値から引き算することで,実験を行う日程や時間帯の変化による温度変化の影響を抑制した.\par

\subsubsection{タスクの評価と成功率}
 折り紙を他に折り目をつけずに教示時と同じ向きに折りたたむことができ,折り紙を自然にしたときに折り目が開かなければ成功とした.10回の試行を行い,9回成功し,成功率は90\%であった.
\par
ここで,重力補償がなければそもそも教示ができないため,成功率は0\%であるといえる.\par

\section{議論}

% 考察・検証
% \subsection{触覚情報の活用による把持成功率の向上}
%  キムタオルの把持タスクにおいて触覚情報の有無による学習モデルの比較検討を行った結果,教示データと同じ指の長さでの把持タスクにおいては,触覚ありモデルの方は教示データに含まれていない高さでも100\%の成功率を達成した.一方,触覚なしモデルでは把持高さが教示データと違うものでは成功率が大きく低下した上,教示データと同じ高さでも一部成功率の低下が確認された.

% この原因として,まず視覚情報の限界があげられうこれはそもそも横からのカメラで高さが分からず,上からのカメラもDepth情報のないRGBカメラであるため,視覚情報だけでは把持高さの正確な推定が困難であったことが原因と考えられる.また,高さによって大きな成功率のばらつきがあるのは,上からのRGBカメラのみによる視覚情報による高さの推定が,うまくできているものとできていないもののばらつきがあり,成功しているものは高さ推定の誤差をそもそもの指の柔らかさが吸収しているためと推察される.このカメラによる誤差を含む情報を触覚情報が補完することで,触覚込みの学習では高さに依らず高い成功率を達成できたと考えられる.
% \par
% Sensor\_off条件において成功率にばらつきが生じた要因として,RGBカメラのみを用いた視覚情報による高さ推定の不安定さが考えられる.トップビューカメラは深度情報を持たないため,積層されたキムタオルの高さ変化を画像特徴の変化として捉える必要があるが,照明条件やわずかな位置ずれにより,その特徴量は不安定となる.実際、Sensor\_off条件での成功エピソードの映像を確認すると,教示データと類似した把持位置にアームが到達していた.一方で失敗エピソードでは,アームが対象の手前で停止,あるいは過剰に押し込む動作が確認された.
% これに対し,Sensor\_on条件では,視覚情報による位置推定に誤差が含まれる場合でも,ソフトフィンガーが対象に接触した際の抵抗値変化をトリガーとして把持動作が調整されていることが示唆される.これは,視覚情報の不確実性を触覚情報が補完し,タスクのロバスト性を向上させた結果であると結論付けられる.\par


% また,教示データと異なる指の長さでの把持タスクにおいても,触覚ありモデルでは高い成功率が維持された.ただ,教示データ時よりも短い指の場合,触覚ありモデルでも1枚のキムタオルの時はつかめなかった.これはそもそもそこまで深くロボットアームを動かすデータが教示データに含まれないため,柔らかさや触覚で適応的に掴む限界がここにあると推察される.触覚なしモデルの方は,指が短くなったことにより,掴みに行く深さが変わり,むしろ多めの枚数で成功が多くみられるため,教示データと同じ指の場合が多めの枚数の時深くつかみに行き過ぎていたと考えられる.\par

% これらの結果から,触覚情報を用いることで視覚情報だけでは捉えきれないコンテクスト情報を補完でき,環境変化に対するロバスト性が向上することが示された.特に,物体の柔らかさや把持高さなどの把持に重要な要素を触覚情報が補完することで,視覚情報だけでは困難な把持タスクにおいても高い成功率を達成できることが分かった.\par

% この模倣学習のアーキテクチャは教示データにない動きからある程度まで逸脱した動きを触覚を入れることによってできるようになるが,教示データの関節角度になく,大きく逸脱したところは,触覚があっても限界がある.これについては,このアーキテクチャの限界であり,教えてない以上のことはそこまでロバストに(結果の右の表の成功率の表のセンサ有での1枚のとき0%のこと)はできないということ



% \subsection{重力補償制御によるタスクの幅の拡大}
%  重力補償がない状態での教示データ収集においては,常にどちらかの手がロボットアームの自重を支えている必要があり,操作しながら考えなければいけないことが増える上,片手での操作はほぼ不可能であった.一方,重力補償制御を導入したことで,ロボットアームの重量を感じさせない無重力のような直感的な操作感が得られ,片手での操作が可能となった.そのうえ,ロボットアームをある位置に固定することが容易になり,固定した状態で人間がロボットアームから手を離してロボットと協働してタスクを補助することができるようになった.それを利用して,今回のような折り紙タスクにおいて,アームに折り紙を折らせて途中で人間が折り紙の固定を補助するという複雑な動作が可能になった.\par
% 定量的に重力補償の有無による操作者の負担等を比較することはできなかったが,実機を通して重力補償の有無を経験することで,教示データ収集時の操作者の主観的な負担が軽減されたことは明らかであった上,操作の幅が広がるということが分かった.\par
\subsection{触覚情報の活用による把持成功率の向上}

 キムタオルの把持タスクにおいて,触覚情報の有無による学習モデルの比較検討を行った.その結果,教示データと同一の指長さを用いた条件において,触覚ありモデル(Sensor\_on)は教示データに含まれない高さ条件であっても100\%の成功率を達成した.一方で,触覚なしモデル(Sensor\_off)では,把持高さが教示データと異なる場合に成功率が著しく低下したほか,教示データと同一の高さであっても一部で成功率の低下が確認された.\par

 Sensor\_off条件において成功率の低下およびばらつきが生じた主要因として,視覚情報のみに依存した高さ推定の限界が挙げられる.本実験環境では,側面カメラは障害物によるオクルージョンが発生し,上面カメラは深度情報を持たないRGBカメラである.そのため,積層されたキムタオルの高さ変化を画像特徴の変化としてのみ捉える必要があるが,照明条件の変動やわずかな位置ずれにより,その特徴量は不安定となる.実際,Sensor\_off条件における失敗エピソードを確認すると,アームが対象の手前で停止する,あるいは過剰に押し込むといった動作が確認された.成功しているエピソードに関しては,偶発的に画像特徴による高さ推定が成功したか,あるいは指の柔軟性が推定誤差を吸収した結果であると推察される.\par

 これに対し,Sensor\_on条件では,視覚情報による位置推定に誤差が含まれる場合でも,高い成功率を維持した.これは,ソフトフィンガーが対象に接触した際の抵抗値変化をトリガーとして把持動作が調整されているためであることが示唆される.すなわち,視覚情報の不確実性を触覚情報が補完することで,環境変動(高さの変化)に対するロバスト性が向上したと結論付けられる.\par

 次に,指の長さを変更した条件(教示時より短い指)での結果について考察する.この条件においても,触覚ありモデルは概ね高い成功率を維持した.しかし,対象物が「1枚(高さ最小)」の場合においては,触覚ありモデルであっても把持に失敗(成功率0\%)した.
 この原因は,教示データに含まれる関節角度の分布にあると考えられる.教示データには,指が短くなった分だけアームを深く下ろすような動作データは含まれていない.本手法で用いたACTアーキテクチャは,触覚情報を利用することで接触時の微調整(局所的な適応)は可能であるが,教示された関節角度の分布から大きく逸脱するような大域的な軌道修正(教示時よりもさらに深くアームを下げる動作)までは生成できない.したがって,1枚の条件での失敗は,学習ベースの手法における汎化性能の限界を示していると言える.\par
 一方で,触覚なしモデルにおいて,指を短くした条件の方がむしろ成功する場合が見られた(多めの枚数時).これは,教示データと同一の指長さでは「深く掴みに行き過ぎていた(過剰な押し込みがあった)」動作が,指が短くなったことで偶然適切な深さになったためと考えられる.\par

 以上の結果より,触覚情報の統合は,物体の柔らかさや接触判定といった視覚のみでは捉えきれないコンテクスト情報を補完し,タスクの成功率向上に大きく寄与することが示された.ただし,その適応能力は教示データの運動学的範囲に依存し,物理的な接触のみでは補いきれない幾何学的な制約が存在することも確認された.

\subsection{重力補償制御によるタスクの幅の拡大}

 本研究で導入した重力補償制御下で,アームの重量を感じさせない直感的な操作感が実現した.さらに特筆すべき点は,アームを任意の位置で静止させ,その姿勢を維持できることである.これにより,操作者が一時的にアームから手を離し,環境側(対象物や治具)への介入を行うといった協働作業が可能となった.

 この利点は,実施した折り紙タスクにおいて顕著に確認された.本タスクでは,「アームによる折り動作」と「人間による治具への固定補助」を交互に行う必要がある.重力補償機能により,操作者はロボットの操作を中断してもその作業状態を維持できるため,一人での教示の幅を大きく拡大できた.\par
 本実験では筋電位計測による作業負荷計測やアンケート手法,教示にかかる時間比較などの定量的な負荷評価は行っていないものの,実機運用を通じて,重力補償の導入が教示者の主観的負担を大幅に軽減し,かつ一人で教示可能なタスクの幅を拡大することを確認した.


\section{結言}

% 結言
% 議論について書くことが多い.様々
%  本研究では,重力補償制御を実装した7自由度ロボットアームのテレオペレーションシステムを構築し,操作者の身体的負担を軽減して直感的で効率的な教示データ収集を可能にした.さらに,指先のソフトセンサから得られる触覚データを学習に取り入れることで,視覚のみでは対応しきれない状況にも適応可能なシステムを構築した.提案手法の有効性を検証するため,紙を折るなどの複雑なタスクを対象に模倣学習実験を行い,重力補償制御と触覚情報の有用性を確認した.\par
%  以上より,本研究は模倣学習で自重の重いロボットアームを使用して複雑なタスクを実行させる際の重力補償制御の重要性と,触覚情報の有効性を示した.

% 本研究で得られた知見は,今後双腕のロボットシステムやより複雑なタスクへの応用が期待される.特に,双腕ロボットシステムにおいては,片方のアームで物体を保持しながらもう片方のアームで操作を行う場面が多く,重力補償制御が操作者の負担軽減に大きく寄与することが予想される.また,触覚情報を活用することで,腕が2本になることにより必然的に増えるセルフオクルージョンの問題を緩和し,より精密な操作が可能になると考えられる.\par

% === 謝辞 ===========================================================
\acknowledgment

本研究の実施や論文の執筆にあたり,多くの方々にご指導とご協力をいただきましたことに,心より感謝申し上げます.京都大学大学院工学研究科機械理工学専攻の細田耕教授,川節拓実講師には,熱心にご指導いただきました.研究室秘書の堀祐子さんには,研究に関わることだけでなく,研究室での生活や様々な場面でお世話になりました.また,研究室の先輩方には様々な助言や激励の言葉をいただき,特に辻大樹さんには,研究に関することについて何度も相談に乗ってくださいました.慣れない環境で初めて研究に取り掛かり,不安なこともたくさんありましたが,研究室同期で話す時間が何より励みになりました.\par
本研究は以上の方々のサポートがなければ終えることは出来ませんでした.この場を借りて深く感謝申し上げます.

\begin{flushright}
令和8年2月 小森 文雄
\end{flushright}


% === 参考文献 =======================================================
\bibliographystyle{junsrt}
\bibliography{bookmark}

% === 付録 ===========================================================
\Appendix
% 付録

\section{使用機材の諸元}
本研究の実験システムを構成するロボットアームおよびカメラの主要な仕様を、表\ref{tab:system_specs}にまとめる。

\begin{table}[htbp]
    \centering
    \caption{実験システムのハードウェア構成}
    \label{tab:system_specs}
    \begin{tabular}{ll}
        \hline
        \textbf{項目} & \textbf{仕様} \\
        \hline
        \multicolumn{2}{l}{\textbf{1. ロボットアーム}} \\
        型番 & RT-CRANE-X7 \\
        自由度 & 7 \\
        作業有効範囲 & 500 mm \\
        可搬重量 & 約 0.5 kg \\
        エンドエフェクタ & 両開きハンド1 \\
        外形寸法 & 130 $\times$ 100 $\times$ 708 mm(固定用金具5mmを含む) \\
        本体重量 & 約 1.5 kg \\
        搭載アクチュエータ & ROBOTIS製 XM540-W270-R, XM430-W350-R \\
        \hline
        \multicolumn{2}{l}{\textbf{2. RGBカメラ}} \\
        型番 & EMEET SmartCam C960 \\
        使用台数 & 2 台 \\
        入力解像度 & 640 $\times$ 480 pixel \\
        フレームレート & 30 fps \\
        画角 (FOV) & 90$^\circ$ \\
        \hline
    \end{tabular}
\end{table}

\subsection{重力補償のキャリブレーションパラメータ}

本研究で導入した重力補償の各関節アクチュエータ(ROBOTIS Dynamixel)におけるトルク電流比および制御パラメータのキャリブレーション結果を参考までに表\ref{tab:gravity_comp_params}に示す。

なお、ベース関節(ID: 2)およびグリッパ部(ID: 8, 9)については、重力補償の適用対象外とした。

\begin{table}[htbp]
    \centering
    \caption{重力補償制御のパラメータ設定}
    \label{tab:gravity_comp_params}
    \begin{tabular}{lc}
        \toprule
        \textbf{対象関節 (ID)} & \textbf{トルク電流換算係数 ($\tau/I$)} \\
        \midrule
        ID: 3 & 0.2267 \\
        ID: 4 & 0.3245 \\
        ID: 5 & 0.3035 \\
        ID: 6 & 0.2545 \\
        ID: 7 & 0.1515 \\
        \bottomrule
    \end{tabular}
\end{table}

\clearpage
\section{キムタオル把持タスクの実験時のロボットの動作遷移}
\setcounter{figure}{0}
\renewcommand{\thefigure}{B.\arabic{figure}}

キムタオル把持タスクにおけるロボットの動作遷移を以下に示す.FigB.1 およびFig B.2 において,同じ記号(\#1~\#8)が付された画像は,側面カメラおよび上面カメラによって同時に撮影されたフレームに対応している.

\begin{figure}[htbp]
  \centering
  % Top row: 4 screenshots
  \begin{subfigure}[t]{0.24\textwidth}
    \centering
    \includegraphics[width=\linewidth]{tex/Image/kimyoko1.pdf}
    \caption*{\#1 Initial position}
  \end{subfigure}\hfill
  \begin{subfigure}[t]{0.24\textwidth}
    \centering
    \includegraphics[width=\linewidth]{tex/Image/kimyoko2.pdf}
    \caption*{\#2}
  \end{subfigure}\hfill
  \begin{subfigure}[t]{0.24\textwidth}
    \centering
    \includegraphics[width=\linewidth]{tex/Image/kimyoko3.pdf}
    \caption*{\#3}
  \end{subfigure}\hfill
  \begin{subfigure}[t]{0.24\textwidth}
    \centering
    \includegraphics[width=\linewidth]{tex/Image/kimyoko4.pdf}
    \caption*{\#4}
  \end{subfigure}

  \vspace{2mm}

  % Bottom row: 4 screenshots
  \begin{subfigure}[t]{0.24\textwidth}
    \centering
    \includegraphics[width=\linewidth]{tex/Image/kimyoko5.pdf}
    \caption*{\#5 Grasping}
  \end{subfigure}\hfill
  \begin{subfigure}[t]{0.24\textwidth}
    \centering
    \includegraphics[width=\linewidth]{tex/Image/kimyoko6.pdf}
    \caption*{\#6 Lifting}
  \end{subfigure}\hfill
  \begin{subfigure}[t]{0.24\textwidth}
    \centering
    \includegraphics[width=\linewidth]{tex/Image/kimyoko7.pdf}
    \caption*{\#7 Releasing}
  \end{subfigure}\hfill
  \begin{subfigure}[t]{0.24\textwidth}
    \centering
    \includegraphics[width=\linewidth]{tex/Image/kimyoko8.pdf}
    \caption*{\#8 Task complete}
  \end{subfigure}

  \caption{Sequence of side-view camera images in time order.}
  \label{figB:kimyoko_timeorder}
\end{figure}

\begin{figure}[htbp]
  \centering
  % Top row: 4 top-view screenshots
  \begin{subfigure}[t]{0.24\textwidth}
    \centering
    \includegraphics[width=\linewidth]{tex/Image/kimue1.pdf}
    \caption*{\#1 Initial position}
  \end{subfigure}\hfill
  \begin{subfigure}[t]{0.24\textwidth}
    \centering
    \includegraphics[width=\linewidth]{tex/Image/kimue2.pdf}
    \caption*{\#2}
  \end{subfigure}\hfill
  \begin{subfigure}[t]{0.24\textwidth}
    \centering
    \includegraphics[width=\linewidth]{tex/Image/kimue3.pdf}
    \caption*{\#3}
  \end{subfigure}\hfill
  \begin{subfigure}[t]{0.24\textwidth}
    \centering
    \includegraphics[width=\linewidth]{tex/Image/kimue4.pdf}
    \caption*{\#4}
  \end{subfigure}

  \vspace{2mm}

  % Bottom row: 4 top-view screenshots
  \begin{subfigure}[t]{0.24\textwidth}
    \centering
    \includegraphics[width=\linewidth]{tex/Image/kimue5.pdf}
    \caption*{\#5 Grasping}
  \end{subfigure}\hfill
  \begin{subfigure}[t]{0.24\textwidth}
    \centering
    \includegraphics[width=\linewidth]{tex/Image/kimue6.pdf}
    \caption*{\#6 Lifting}
  \end{subfigure}\hfill
  \begin{subfigure}[t]{0.24\textwidth}
    \centering
    \includegraphics[width=\linewidth]{tex/Image/kimue7.pdf}
    \caption*{\#7 Releasing}
  \end{subfigure}\hfill
  \begin{subfigure}[t]{0.24\textwidth}
    \centering
    \includegraphics[width=\linewidth]{tex/Image/kimue8.pdf}
    \caption*{\#8 Task complete}
  \end{subfigure}

  \caption{Sequence of top-view camera images in time order.}
  \label{figB:kimue_timeorder}
\end{figure}



\clearpage
\section{折り紙タスクの実験時のロボットの動作遷移}


\clearpage
\section{触覚を融合したACTモデルの詳細}
特徴抽出と埋め込み
本システムには,RGBカメラ画像、ロボットの関節角度、および指先触覚センサの値が入力される。Transformerは内部計算において、全ての入力データが同一の次元(ベクトル長)を持つことを前提とするため、各モダリティ(情報の種類)に対して以下の前処理を行い、共通の次元 $d_{model}$ を持つ「トークン(特徴ベクトル)」へ変換する。\par

画像情報のエンコード: 複数のカメラから得られたRGB画像は、学習済みの畳み込みニューラルネットワーク(CNN)である ResNet18 等に入力される。CNN は画像からエッジやテクスチャなどの空間的な特徴を抽出する。得られた特徴マップは平坦化(Flatten)された後、線形層(Linear layer)を介して次元 $d_{model}$ へと射影される。\par

状態情報のエンコード: 関節角度や触覚センサの値といった低次元の物理量は、多層パーセプトロン(MLP)等の線形層によって、同様に次元 $d_{model}$ へと射影される。これにより、画像のような高次元データと、センサ値のような低次元データを同一の特徴空間で扱うことが可能となる。\par

位置埋め込み(Positional Embedding): Transformer 自体はデータの入力順序を認識できない構造であるため、各トークンに対して系列内の位置情報を表すベクトル(位置埋め込み)を加算する。これにより、モデルは「どのデータが何番目の時刻のものか」を区別できるようになる。\par

学習フェーズ
学習時の目的は、与えられた状況(観測)に対してエキスパート(人間)がどのような意図やスタイルで動作したかをモデルに学習させることである。学習時には CVAE エンコーダへ現在の観測と教師データである未来のアクション系列を入力する。\par

[CLS] トークンによる情報の集約: CVAE エンコーダには、入力系列の先頭に [CLS](Classification Token)と呼ばれる特殊な学習可能トークンを付加する。Transformer の自己注意機構により各トークンは互いの情報を参照し合うが、[CLS] トークンは系列全体の情報を自身に集約するよう学習され、結果として [CLS] トークンは「この観測状況においてどのようなアクションが行われたか」というエピソード全体の特徴(スタイル)を圧縮した表現となる。\par

潜在変数 $z$ の生成: エンコーダの出力から [CLS] トークンに対応するベクトルを抽出し、これを線形層で変換して平均 $\\mu$ と分散 $\\sigma$ を算出する。これらをパラメータとする正規分布から潜在変数 $z$ をサンプリングする。この $z$ は,各状況における動作の「確率的変動要素(例:速度やスタイルの違い)」を表現する。\par

推論フェーズ
推論時および学習時の再構成では、Transformer のエンコーダ・デコーダ構造を用いて、観測データと潜在変数 $z$ から未来のアクションを予測する。\par

観測データの統合(Encoder): 現在の観測データ(画像・関節・触覚)と CVAE から得られた潜在変数 $z$ を結合し、Transformer エンコーダに入力する。ここで Self-Attention により、例えば「画像内の物体の位置」と「現在の腕の角度」の関係性を紐付けるようなモダリティ間の統合処理が行われる。\par

アクション系列の生成(Decoder): Transformer デコーダにおけるクエリとして、予測したい未来のステップ数に対応する固定位置埋め込み(Fixed Position Embeddings)を与える。デコーダは Cross-Attention を用いてエンコーダからの情報(Key/Value)を参照しながら各タイムステップのアクション値を算出することで、現在から未来 $k$ ステップ分のアクション系列を一括して出力する。\par

このように、[CLS] トークンを用いた情報の圧縮(CVAE)と Attention 機構によるマルチモーダル情報の統合(Transformer)を組み合わせることで、視覚・固有感覚・触覚を高度に融合したロバストな動作生成が可能となる。

\end{document}